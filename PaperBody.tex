\section{Introduction}


\section{Related Work}
The Next Release Problem(NRP)\cite{NRP} was proposed by Bagnall et al in 2001. In this paper, authors considered requirements which can be dependent on other requirements. And also, they set the importance(or weight) of each customer who demands for resolving requirements to company.

In the Multi-Objective Next Release Problem(MONRP)\cite{MONRP}, authors tried to solve this problem as the multi-objective optimization problem. They applied algorithms(Pareto GA, Single-objective GA, and NSGA-II) and concluded that the NSGA-II is well suited to the MONRP.

After these papers, there are many trials to select the optimal set of requirements.\cite{ILP} \cite{DE} \cite{ACO} However, in almost cases, they used the data set which is randomly generated.

Tonella et al. were applied the Interactive Genetic Algorithm and AHP to the NRP and requirements prioritization for real software system as part of the project ACube, but their constraint was number of information about the priority graph, not the limited budget.\cite{IGA} 



\section{The Model}

Let $R = \{r_1, r_2, \ldots, r_N\}$ is a set of all possible N requirements for software or company and each requirement $r_i \in R$ has its cost and profit, $cost_i \in \mathbb{Z}^+$ and $profit_i \in \mathbb{Z}^+$, to be performed. The cost is whatever company have to pay to resolve requirements, for example, money, time and etc. The profit is whatever company get by resolving requirements, for example, user satisfaction. In this project, we assumed that all requirements are independent. It means, there is no prerequisite of requirements. The cost vector $Cost$ and the profit vector $Profit$ for requirements are denoted by 
\[
Cost = \{cost_1, cost_2, \ldots, cost_N\}
\]
\[
Profit = \{profit_1, profit_2, \ldots, profit_N\}
\]

The company always wants to select requirements for the next release to get the highest sum of profits in the limited budget. Let the decision vector $x = \{x_1, x_2, \ldots, x_n\} \in \{0, 1\}$ where $x_i$ is 1 if requirement $i$ is selected and 0 otherwise. Then our objective function can be written as below:
\[
maximize \sum_{i = 1}^{N} profit_i \cdot x_i
\]
subject to
\[
\sum_{i = 1}^{N} cost_i \cdot x_i \leq B
\]
for some bound $B \in \mathbb{Z}^+$ which represents the limited budget.



\section{Algorithms}
We applied two deterministic algorithms(AHP and ILP) and three heuristic algorithms(SA, GA and NSGA-II) which were introduced as the algorithms well-suited for the NRP in papers.\cite{NRP}\cite{ILP}\cite{IGA}\cite{MONRP}


\subsection{Analytic Hierarchy Process}

\subsection{Integer Linear Programming}
Integer Linear Programming(ILP) is optimization method whose objective function and constraints are linear, and all variables are integer. Reference paper used ILP solver named \textit{CPLEX}, but it is commercial solver so we used \textit{PuLP}(modeler) and \textit{Gurobi}(solver) for this project. \textit{PuLP} is free open source software so users can use it by pip install. However general \textit{Gurobi} solver is commercial one so we registered to \textit{Gurobi (www.gurobi.com)} and used \textit{Gurobi} academic version.

\subsection{Simulated Annealing}
Simulated annealing(SA) is one of the local search techniques which are iterative in nature and rely on the definition of a solution neighbourhood. For our NRP project, a requirement-based neighbourhood was defined. So the current solution can reach its neighbour by making small changes of decision vector $x$.

To apply the SA, we have to define a objective function to get the optimal decision vector:
\[
f(\textbf{x}) = \sum_{i = 1}^{N} profit_i \cdot x_i + \lambda \min \Big\{0, \sum_{i = 1}^{N} (cost_i \cdot x_i) - B \Big\}
\]


for NRP was introduced in the NRP paper by Bagnall et al.\cite{NRP}


In this paper, the cooling schedule by Lundy and Mess\cite{LundySA} is well-suited. Their approach is presented below:
\[
    T_{i+1} = \frac{T_i}{1 + \beta T_i}
\]
where $T_i$ is the temperature at iteration $i$ and $\beta$ is control parameter. So for the Lundy and Mess cooling schedule, the temperature drops after each move.

In our project, the parameter $\beta$ was set to $1 \times 10^{-8}$ since Bagnall et al. got the best solution for this value in their paper.

\subsection{Genetic Algorithm}

\subsection{NSGA-II}

\begin{algorithm}
\caption{NSGA-II}\label{alg:nsga2}
\begin{algorithmic}
    \While{not stopping condition}
        \State Let $R_t = P_t \cup Q_t$ ;
        \State Let $F = $ fast-non-dominated-sort ($R_t$) ; %\Comment{}
        \State Let $P_{t+1} = \emptyset$ and $i = 1$ ;
        \While{$|P_{t+1}| + |F_i| \leq N$}
            \State Apply crowding-distance-assignment($F_i$) ; %\Comment{}
            \State Let $P_{t+1} = P_{t+1} \cup F_i$ ;
            \State Let $i = i + 1$
        \EndWhile
        \State $Sort(F_i, \prec_n)$
        \State Let $P_{t+1} = P_{t+1} \cup F_i[1:(N - |P_{t+1}|)]$ ;
        \State Let $Q_{t+1} = $ make-new-pop($P_{t+1}$) ;
        \State Let $t = t + 1$ ;
    \EndWhile\label{euclidendwhile}
\end{algorithmic}
\end{algorithm}

\begin{table*}
  \caption{Profit Comparison}
  \label{tab:commands}
  \begin{tabular}{cccccl}
    \toprule
    &eclipse-log&firefox-log&firefox-priority&firefox-comments+priority\\
    \midrule
    Real Cost&50699.70190&262945.49581&262945.49581&262945.49581 \\
    Real Profit&3489.63592&13025.88085&12210&26332.88085 \\
    \hline
    SA&3702.85182&13524.93411&16300.0&28358.2295685 \\
    GA&3793.5045&13662.38600&16303.0&28522.11861 \\
    NSGA-II&3822.47741&13700.52067&16518.33333&28729.33476 \\
    ILP&3825.42768&13768.44722&16336.0&28845.23124 \\
    AHP&3810.3125825&13758.5577962&16249.0&28811.2749907 \\
    \bottomrule
  \end{tabular}
\end{table*}
% end the environment with {table*}, NOTE not {table}!

\begin{figure*}[h]
\begin{subfigure}{\columnwidth}
\includegraphics[width=0.9\linewidth]{bugzilla_eclipse_log(comments)_2016meancost.JPG}
\caption{bugzilla-eclipse-log(comments)-2016meancost}
\label{fig:subim1}
\end{subfigure}
\begin{subfigure}{\columnwidth}
\includegraphics[width=0.9\linewidth]{bugzilla_firefox_log(comments)_2016meancost.JPG}
\caption{bugzilla-firefox-log(comments)-2016meancost}
\label{fig:subim2}
\end{subfigure}
\begin{subfigure}{\columnwidth}
\includegraphics[width=0.9\linewidth]{bugzilla_firefox_priority_2016meancost.JPG}
\caption{bugzilla-firefox-priority-2016meancost}
\label{fig:subim3}
\end{subfigure}
\begin{subfigure}{\columnwidth}
\includegraphics[width=0.9\linewidth]{bugzilla_firefox_comments-priority_2016meancost.JPG}
\caption{bugzilla-firefox-comments+priority-2016meancost}
\label{fig:subim4}
\end{subfigure}

\caption{Profit Comparison for each dataset}
\label{fig:image1}
\end{figure*}

\begin{figure*}[h]
\begin{subfigure}{\columnwidth}
\includegraphics[width=0.9\linewidth]{SA-correlation.JPG}
\caption{SA}
\label{fig:subim5}
\end{subfigure}
\begin{subfigure}{\columnwidth}
\includegraphics[width=0.9\linewidth]{GA-correlation.JPG}
\caption{GA}
\label{fig:subim6}
\end{subfigure}
\begin{subfigure}{\columnwidth}
\includegraphics[width=0.9\linewidth]{ILP-correlation.JPG}
\caption{ILP}
\label{fig:subim7}
\end{subfigure}
\begin{subfigure}{\columnwidth}
\includegraphics[width=0.9\linewidth]{AHP-correlation.JPG}
\caption{AHP}
\label{fig:subim8}
\end{subfigure}

\caption{Cost-Profit Correlation}
\label{fig:image2}
\end{figure*}

\begin{figure*}[h]
\includegraphics[width=0.9\linewidth]{cost-profit-together.JPG}
\caption{Cost-Profit Correlation in one graph}
\label{fig:image3}
\end{figure*}

\section{Results}
At first experiment, we applied five algorithms to four datasets and compared the maximum profit each other. In \textbf{table 1}, upper two rows are real cost sum and real profit sum of each datasets. Below these rows, there are maximum profits for each algorithms in four datasets. At the result, every maximum profits are bigger than real profit sum. Also, consumed time to return maximum profit was exceedingly short in ILP and AHP, rather than SA, GA, and NSGA-II. \\
\textbf{Figure 1} is the graphs of \textbf{table 1}. Each graph expresses the maximum profit of five algorithms. At the result, the maximum profit of ILP is the biggest, and the maximum profit of NSGA-II and AHP is bigger than the maximum profit of SA and GA in most case. Except ILP and AHP, maximum profit is high in order of NSGA-II, GA, then SA. Peculiar thing is that, in dataset \textit{bugzilla-firefox-priority-2016meancost}, NSGA-II is largely bigger than other four algorithms. The reason we suppose is that other three datasets' profit is variety because the number of comments is not bounded, but this dataset's profit is priority, which is varied only in 1, 2, 3, 4, and 5 points. In this case, requirements can be seen more similar each other than other datasets. This difference could influence the difference of the result.\\
At experiment 2, we applied four algorithms except NSGA-II to dataset \textit{bugzilla-firefox-comments+priority-2016meancost}. At this experiment, we got the maximum profits while changing the cost constraints. The result is expressed in \textbf{Figure 2} and \textbf{Figure 3}. \textbf{Figure 2} expresses maximum profits on each cost, and \textbf{Figure 3} is integrating graph of \textbf{Figure 2} because of comparing the difference for four algorithms. \\
The result is that they are all increasing almost linearly and similarly. One of the reasons to perform this experiment was that we wondered if there is a peculiar part in cost-profit graph.(if there is exceeding increase in profit) And at the result, there is no prominently peculiar part, and all the results are increasing normally.
\section{Conclusion}




